\documentclass{jdmdh}
\usepackage{array}
\usepackage{pgfplots}

\title{From me to you: peer-to-peer collaboration with linked data}
\author{Jan Kaßel}
\author{Dr. Thomas Köntges}
\affil{Leipzig University, Germany} 

\corrauthor{Jan Kaßel}{jan.kassel@studserv.uni-leipzig.de}

\begin{document}

\maketitle

\abstract{In recent years, Digital Humanities’ collaborative nature has caused an awakening of digitally native research practice, where interdisciplinary workflows commonly feed into centralized data repositories. Connecting these repositories, the W3C’s Web Annotation specification builds upon linked data principles for targeting any web resource or linked data entity with syntactic and semantic annotation. However, today’s platform-centric infrastructure diminishes the distinction between institutions’ and individuals’ data. This poses issues of digital ownership, interoperability, and the privacy of data stored on centralized services. With Hyperwell, we aim to address these issues by introducing a novel architecture that offers real-time, distributed synchronization of web annotations, leveraging contemporary peer-to-peer technology. Extending the peer-to-peer network, institutions provide Hyperwell gateways that bridge peers’ annotations and the web. These gateways affirm a researcher’s affiliation, acting as a mere mirror of that researcher’s data, while maintaining digital ownership.}

\keywords{linked data; peer-to-peer systems; annotation; real-time collaboration}

% twist: load markdown paper
\markdownInput{paper.md}

\bibliographystyle{plainnat}
\bibliography{references.bib}

\end{document}