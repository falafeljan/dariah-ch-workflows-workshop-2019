\documentclass{jdmdh}
\usepackage{array}
\usepackage{pgfplots}

\title{Lorem ipsum dolor sit amet, consectetur adipiscing elit}
\author{Jan Kaßel}
\author{Dr. Thomas Koentges}
\affil{Leipzig University, Germany} 


\begin{document}

\maketitle

\abstract{In recent years, Digital Humanities' collaborative nature caused a wake digitally-native research practice, where interdisciplinary workflows commonly feed into centralized data repositories. Connecting these repositories, the W3C’s Web Annotation specification builds upon Linked Data principles for targeting any web resource or Linked Data entity with syntactic and semantic annotation. However, today’s platform-centric infrastructure diminishes the distinction between institutional and individuals’ data. This poses issues of digital ownership, interoperability, and privacy of data stored on centralized services. With Hyperwell, we aim to address these issues by introducing a novel architecture that offers real-time, distributed synchronization of Web Annotations, leveraging contemporary Peer-to-Peer technology. Extending the Peer-to-Peer network, institutions provide Hyperwell gateways that bridge peers’ annotations into the common web. These gateways affirm a researcher’s affiliation while acting as a mere mirror of researchers’ data and maintaining digital ownership.}

\keywords{linked data; peer-to-peer systems; data ownership; distributed systems; annotation; real-time collaboration}

% twist: load markdown paper
\markdownInput{paper.md}

\bibliographystyle{plainnat}
\bibliography{refs}

\end{document}